% Created 2020-09-11 Fri 13:27
% Intended LaTeX compiler: pdflatex
\documentclass[11pt]{article}
\usepackage[utf8]{inputenc}
\usepackage[T1]{fontenc}
\usepackage{graphicx}
\usepackage{grffile}
\usepackage{longtable}
\usepackage{wrapfig}
\usepackage{rotating}
\usepackage[normalem]{ulem}
\usepackage{amsmath}
\usepackage{textcomp}
\usepackage{amssymb}
\usepackage{capt-of}
\usepackage{hyperref}
\author{Sudhan Chitgopkar}
\date{\today}
\title{}
\hypersetup{
 pdfauthor={Sudhan Chitgopkar},
 pdftitle={},
 pdfkeywords={},
 pdfsubject={},
 pdfcreator={Emacs 26.3 (Org mode 9.1.9)}, 
 pdflang={English}}
\begin{document}

\tableofcontents

\section{09.11.20}
\label{sec:org9e9a902}
\begin{itemize}
\item Northern latitudes experience greater seasonality in CO2 concentrations
\begin{itemize}
\item This is due to variation in photosynthetic activity by plants
\end{itemize}
\item Greenhouse effect
\begin{itemize}
\item Some incoming solar radiation is absorbed
\item Other amounts are reflected back into the atmosphere
\item Greenhouse gases capture and reradiate some heat over and over, warming the Earth
\item More gases, more heat
\end{itemize}
\item Albedo: measure of the reflectivity of a surface
\begin{itemize}
\item light surfaces have a higher albedo, darker surfaces have a lower albedo
\item surfaces with a low albedo release more heat into the atmosphere
\end{itemize}
\item Positive Feedback Loops
\begin{itemize}
\item applied to albedo:
\item temps rise -> more ice melting -> more water warming -> temps rise
\end{itemize}
\item Urban Heat Island Effect
\begin{itemize}
\item cities will be inc their population, inc energy and temperature
\item cities in particular have higher temperatures
\item tree cover -> cooler temperatures
\end{itemize}
\item Small changes in overall global temp can cause significant changes
in weather creating more extreme storms and more record temps
\begin{itemize}
\item roughly twice as many heat records
\item alterations in global jet streams
\item frost comes later and begins earlier
\end{itemize}
\item General climate change impacts:
\begin{itemize}
\item Health impacts
\item Crop productivity
\item Coastal erosion
\item Biodiversity
\item Water availability
\item Fire risk
\end{itemize}
\item Weather events getting more extreme with
\begin{itemize}
\item sea levels
\item wildfires
\end{itemize}
\item Need both adaptation and mitigation
\begin{itemize}
\item adaptation: responding to warming that has already happened
\item mitigation: preventing further warming by addressing climate change causes
\end{itemize}
\end{itemize}
\section{09.09.20}
\label{sec:orgf9ddda8}
\subsection{The Earth's Atmoshphere}
\label{sec:org58ea9f6}
\begin{itemize}
\item Climate change is a serious environmental problem impacting species, ecosystems, and the globe
\item The atmosphere helps protect the Earth from the sun and keeps the temperature of the Earth cool
\item Atmosphere has a significant impact on climate
\item Earth's Atmosphere Composition
\begin{itemize}
\item Nitrogen (78\%)
\item Oxygen (21\%)
\item Other - Greenhouse Gases (1\%)
\end{itemize}
\end{itemize}
\subsection{The Keeling Curve}
\label{sec:org517d6cd}
\begin{itemize}
\item 
\end{itemize}
\section{09.02.20}
\label{sec:org656584c}
\subsection{Demographic Transition Model}
\label{sec:org50d3079}
\begin{itemize}
\item Demographers use age structure diagrams to predict future growth potential of a population
\begin{itemize}
\item Pyramid structures indicate fast growth
\item House-shaped structures have moderate growth
\item Diamond structures have low/negative growth
\end{itemize}
\item Development leads to smaller families
\item Demographic transitions happen country by country
\item Industrialization might not lead to a demographic transition in all countries
\begin{itemize}
\item May not be linked to quality of life
\item Religion/Cultural beliefs
\item Social justice issue, improving the well-being of women and children key to dec. fertility
\end{itemize}
\end{itemize}
\subsection{Social Justice: Education for Women}
\label{sec:org12fd2b8}
\begin{itemize}
\item Education of girls \& economic opportunities for women are correlated with lower birth rates
\item Education empowers women to take control over thri own fertility through: 
\begin{itemize}
\item Birth control
\item Marrying later
\item Delaying childbirth for career opportunities
\end{itemize}
\item Women earning more money is correlated to lower child mortality
\end{itemize}
\subsection{Environmental Impact}
\label{sec:orgea7a777}
\begin{itemize}
\item Slowing population growth is critical to sustainability and reducing our population impact
\item Our impact on the population is a result of (1) our population size and
(2) our consumption habits - both must be addressed
\item Ecological footprint: the land area needed to provide the resources for, and assimilate
the waste of, a person or population
\end{itemize}
\subsection{Sustainability}
\label{sec:org669ebcc}
\begin{itemize}
\item A dynamic process between the economy, society, and environment
\item Sustainable: The process or the activity can be mantained without exhaustion or collapse
\begin{itemize}
\item Intra \& Inter-generational issue
\item Capacity of a system to accomodate changes:
\begin{itemize}
\item rates of renewable resource use should not exceed regeneration rate
\item rates of non-renewable resource use should not exceed rate of renewable substitute dev
\item rates of pollution should not exceed ssimilative capacity of the environment
\end{itemize}
\end{itemize}
\item Sustainable development has three factors:
\begin{itemize}
\item Social equity
\item Economic efficiency
\item Environmental responsibility
\end{itemize}
\end{itemize}
\subsection{Worldviews}
\label{sec:org0bff237}
\begin{itemize}
\item Culture influences our beliefs through:
\begin{itemize}
\item Knowledge
\item Beliefs
\item Values
\item Learned ways of life
\end{itemize}
\item Worldviews are affected by: 
\begin{itemize}
\item Environmental Ethics
\end{itemize}
\end{itemize}
\section{08.31.20}
\label{sec:orgf289b35}
\subsection{Human Populations}
\label{sec:org1d8c731}
\begin{itemize}
\item 3 major sparks of growth
\begin{itemize}
\item Agricultural Revolution
\item Industrual Revolution
\item Green Revolution
\end{itemize}
\item With more food and technology, the population and need for more human labor increased
\item The human population is rapidly increasing and the impact of humans is due to:
\begin{itemize}
\item More humans overall
\item Greater growth / person
\end{itemize}
\item To address population growth, we need to pursue a variety of approaches that address factors
encouraging high birth rates
\item Zero population growth: the absence of population growth, occurs when birth rates = death rates
\begin{itemize}
\item Replacement fertility is reached
\end{itemize}
\end{itemize}
\subsection{Population Ecology}
\label{sec:org70116b8}
\begin{itemize}
\item Analyze and categorize human populations using population ecology techniques
\item Population Ecology: a branch of biology dealing with the number of individuals
in a particular species in an area over time
\item Ecologists study populations to understand what makes them survive and thrive
\item Size, distribution, and growth rate is influenced by a variaty of factors and are important to 
understanding popilation ecology
\end{itemize}
\subsection{Monitoring Population Dynamics}
\label{sec:orgcaa5f4c}
\begin{itemize}
\item Population Dynamics: Changes over time in population size and composition
\item Important metrics:
\begin{itemize}
\item Minimum viable population - min number of individuals that would still allow population to persist or grow
\item Carrying Capacity (K) - the maximum population size that a particular environment can support indefinitely
\end{itemize}
\item Population Density - the overall desnity a particular populaiton can sustain
\end{itemize}
\subsection{Exponential Growth \& Populations}
\label{sec:orgdf7bb65}
\begin{itemize}
\item Exponential growth occurs in populations when growth is unrestricted. This is, overall, unsustainable
\item Growth which becomes progressively larger each breeding cycle
\item Produces a J curve when plotted
\end{itemize}
\subsection{Monitoring Population Growth}
\label{sec:orgcebf02b}
\begin{itemize}
\item Population growth rate - the rate at which a population of a species grows over time
\item Growth factors - factos which assist in the growth of a population
\item Resistance factors - factors which inhibit the growth of a population
\item Limiting factos: resources needed for survival but that may be in short supply
\end{itemize}
\subsection{Logistic Growth}
\label{sec:org0412553}
\begin{itemize}
\item Occurs when a population nears carrying capacity (k) 
\begin{itemize}
\item Maximum sustainable population size
\item Determined by limiting factors
\end{itemize}
\end{itemize}
\subsection{Density-dependent/ Density-independent Factors}
\label{sec:orge501453}
\begin{itemize}
\item Density dependent factors increase as populations grow, typically biotic
\begin{itemize}
\item Disease
\item Competition
\item Predation
\end{itemize}
\item Density independent facts affect population growth regardless of population size
\begin{itemize}
\item Storm
\item Fire/Flood
\item Avalanche
\end{itemize}
\end{itemize}
\subsection{Regulation}
\label{sec:org939a009}
\begin{itemize}
\item Tendency for populations to decrease in size when above acertain level, and increase
in size below that level
\item Populations can only be regulated by density-dependent factors
\item Top down Regulation
\begin{itemize}
\item Predation
\item Disease
\end{itemize}
\item Bottom up Regulation
\begin{itemize}
\item Nutrients
\item Water
\item Sunlight
\end{itemize}
\end{itemize}
\section{08.28.20}
\label{sec:org02388fb}
\subsection{What is Science?}
\label{sec:orgb53070e}
\begin{itemize}
\item Science: a body of knowledge that allows us to understand the world around us
\item Science is based on empirical evidence
\item Science allows us to test our ideas and evaluate the evidence
\item Scientific knowledge, including facts, theories, and laws, is subject to change
\item Scientific claims change as new evidence is made available
\end{itemize}
\subsection{White-Nose Syndrome Case Study}
\label{sec:orgbc955db}
\subsubsection{About WNS}
\label{sec:org1fe1f1a}
\begin{itemize}
\item White-Nose Syndrome
\begin{itemize}
\item 2007-2016, 6+ million bats dead as a result of White Nose Syndrome
\item The reason for the deaths was White-Nose Syndrome
\end{itemize}
\item Chytridiomycosis
\begin{itemize}
\item Infectious, fungal disease affecting amphibians
\item Helped understand white-nose syndrome with bats
\end{itemize}
\end{itemize}
\subsubsection{Science with WNS}
\label{sec:org5e0b9e5}
\begin{itemize}
\item Scientific Method: the procedure used to empirically test a hypothesis
\begin{enumerate}
\item Observations generate questions
\item Choose a question to investigate
\item Consult literature
\item Develop a hypothesis and make a testable prediction
\item Design and carry out a study
\item Analyze data
\item Draw a conclusion
\end{enumerate}
\item Inferences: Conclusions drawn based on observations
\item Hypothesis: An inference that proposes possible explanation that includes previous knowledge/observation
\item Testing a Hypothesis: Hypotheses can be tested through an observational or experimental study
\item Scientific Studies: A fair test with results that could support or falsify the research prediction
\begin{itemize}
\item Experimental Studies: Conditions are manipulated intentionally
\begin{itemize}
\item Test Group: the group in an experimental study such that it differs from the control in only one way
\item Control Group: the group in an experimental study to which the test group's results are compared
\end{itemize}
\item Observational Studies: Gather real-world data without any intentional variable manipulation
\end{itemize}
\item Theory: A hypothesis that survives repeated testing by significant research can become a theory
\item Correlation v Causation
\begin{itemize}
\item Correlation: two things occuring together but not necessarily having a cause-effect relationship
\item Cause-Effect Relationship: the associationof a two variables that identifies one variable occurring
as a result of the other
\item Observational studies can derive correlation but not causation
\item Experimental studies can derive causational relationships
\end{itemize}
\item Policy: a formalized plan that addresses a desired outcome or goal
\begin{itemize}
\item policies need to be flexible, adapt to new findings, address the environmental problem, fit social need
and be economically viable in order to work effectively.
\end{itemize}
\end{itemize}
\subsection{Summary}
\label{sec:org28b2956}
\begin{itemize}
\item Scientific knowledge, through reliable and durable, is never absolute pr certain
\item This knowledge, including facts, theories, and laws, is subject to change
\item Physical evidence, systematically collected and logically analyzed, helps scientists
understand environmental issues and guide policy decisions
\end{itemize}
\section{08.25.20}
\label{sec:org8cebc09}
\subsection{Applied v Empirical Science}
\label{sec:org9665ef0}
\begin{itemize}
\item Applied Science = research whose findings are used to solve practical problems
\item Empirical science: A scientific approach that investigates the natural world through case studies
\end{itemize}
\subsection{Social Traps}
\label{sec:org1e2908b}
\begin{itemize}
\item Occurs when a large amount of people are using a shared resource
\item Seem good in the short term but are actually bad in the long term
\item 3 Types:
\begin{itemize}
\item Tragedy of the Commons: When resources are shared, individuals try to maximize personal
benefit which hurts the resource itself
\item Time delay: Collective decisions that are good today but gone tomorrow
\item Sliding reinforcer: related to the evolution of natural organisms and GMOs
\end{itemize}
\end{itemize}
\subsection{Beginning with Data Interpretation}
\label{sec:org3e8d681}
\begin{itemize}
\item Variables represent factors that can be manipulated, controlled, or merely measured for research
\item Variation = how much a variable changes
\item Independent var is controlled to see effects in the Dependent var
\item Graphs explore relationships with data and report this data
\end{itemize}
\subsection{Observational v Experimental Studies}
\label{sec:orgba1680a}
\begin{itemize}
\item Observational studies can observe a correlation but are unable to derive a causational reln.
\item Experimental studies have a control var (required) and are able to derive causactional rlns.
\end{itemize}
\section{08.24.20}
\label{sec:orgd45a5f8}
\subsection{Definitions}
\label{sec:org3de021d}
\begin{itemize}
\item Ecology: the branch of science dealing with the relationships of living things to one another \& the environment
\item Environmental Science: The study of all aspects of the environment, including physical, chemical, and biological factos, particularly with respect to how these aspects affect humans, and vice versa
\item Environmental Ethics: Personal philosophy that influences how a person interacts with their natural environment and thus influences how one responds to environmental problems
\end{itemize}
\subsection{Ecology != Environmentalism}
\label{sec:orgbac2735}
\begin{itemize}
\item Distinguish between envrironmentalism \& ecology
\end{itemize}

\begin{center}
\begin{tabular}{ll}
Environmentalism & Ecology\\
\hline
Activism to protect the environment & Scientific study of living and non-living things\\
\end{tabular}
\end{center}
\end{document}