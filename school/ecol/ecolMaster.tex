% Created 2020-10-08 Thu 20:53
% Intended LaTeX compiler: pdflatex
\documentclass[11pt]{article}
\usepackage[utf8]{inputenc}
\usepackage[T1]{fontenc}
\usepackage{graphicx}
\usepackage{grffile}
\usepackage{longtable}
\usepackage{wrapfig}
\usepackage{rotating}
\usepackage[normalem]{ulem}
\usepackage{amsmath}
\usepackage{textcomp}
\usepackage{amssymb}
\usepackage{capt-of}
\usepackage{hyperref}
\author{John Doe}
\date{\today}
\title{}
\hypersetup{
 pdfauthor={John Doe},
 pdftitle={},
 pdfkeywords={},
 pdfsubject={},
 pdfcreator={Emacs 27.1 (Org mode 9.4)}, 
 pdflang={English}}
\begin{document}

\tableofcontents

\section{10.07.20}
\label{sec:orga1c30e1}
\subsection{Genetic Diversity \& Natural Selection}
\label{sec:org3568b12}
\begin{itemize}
\item Genetic diversity in a population is the raw material natural selection
\item The larger the amount of genetic diversity, the higher probability that some individuals from
that pool can survive changes to its environment
\item Phenotype = expressed gene
\item Natural selection acts directly on the phenotype, resulting in changes in allele frequencies
from parental to offspring generations
\end{itemize}
\section{10.05.20}
\label{sec:orge50cb91}
\begin{itemize}
\item Following widespread usage of antibiotics on humans and animals, waste from livestock and humans
is generating antibiotic-resistance bacteria
\item These bacteria are getting back into the environment through out waste
\end{itemize}
\subsection{Antibiotic Resistance is:}
\label{sec:org2895b1b}
\begin{itemize}
\item A complex problem that involves helping many actors see the big picture and not just their
part of it
\item Issues where an action affects (or is affected by) the environment surrounding the issue,
either the natural environment or the competitive environment
\item Problem whose solutions are not Obvious
\end{itemize}
\subsection{Systems Thinking}
\label{sec:orge807285}
\begin{itemize}
\item Considers the whole rather than parts of the whole:
\begin{itemize}
\item Events
\item Patterns
\item Underlying Structure
\end{itemize}
\end{itemize}
\subsection{Cycle of Infection}
\label{sec:orge7c71b1}
\begin{itemize}
\item Farm animals recieve antibiotics often, developing resistant bacteria in their gut
\item This can be transmitted through produce, waste, shared environments, etc.
\end{itemize}
\subsection{Bacteria}
\label{sec:orgab3b0f3}
\begin{itemize}
\item Bacteria are single celled organisms that can grow in colonies
\item Many different kinds of bacteria can grow together in similar environments
\end{itemize}
\subsection{Explaining Resistance}
\label{sec:org6e04aa5}
\begin{itemize}
\item Antibiotics kill almost all antibiotic sensitive bacteria, leaving few sensitive and many unsensitive
\item Reproduction occurs with the mostly-unsensitive remaining bacteria, leaving to many unsensitive off-
spring. This increases the amonut of resistant bacteria as a whole.
\end{itemize}
\subsection{Genetic Variation}
\label{sec:orga97e236}
\begin{itemize}
\item Variation in the susceptability of bacteria to antibiotics allows for the propogation of 
these genes in bacterial communities
\item Individuals of the same species have the same basic gene
\item Alleles: variants of genes that account for the diversity of traits seen in a populat
\item Adaptation: traits that promote the success of a species
\item An adaptive trait for one environmental condition does not mean that it is adaptive for all conditions
\end{itemize}
\subsection{Genetic Diversity}
\label{sec:orge87fbc8}
\begin{itemize}
\item Within populations, biodiversity is measured by genetic diversity
\item Genetic diversity improves survival of a population
\item Outbreeding, through sexual reproduction of not closely related individuals, maximizes genetic 
diversity
\item Inbreeding, or mating between closely related individuals, results from small 
populations, and increases chances of genetic diseases (e.g., hemophilia, cystic fibrosis, etc.)
\end{itemize}
\subsection{Sources of Genetic Variation}
\label{sec:org1f2fc9d}
\begin{itemize}
\item Mutation: A change in the DNA sequence of sex cells that alter a gene
\begin{itemize}
\item Can be neutral, beneficial, or harmful
\end{itemize}
\item Genetic Recombination: The production of eggs and sperm that results in a shuffling of 
alleles, creating new combinations in offspring
\end{itemize}
\subsection{Natural Selection}
\label{sec:org06f11a8}
\begin{itemize}
\item Constant struggle of organisms to survive and mate
\item Organisms tend to produce more offspring that can survive
\item Individuals of the same species are not identical
\item Evidence of Natural Selection: Selective breeding (artificial selection) of dogs and cats
\item Natural selection results in changes in gene frequencies
\begin{itemize}
\item Some individuals will be able to obtain more resources and can produce more offspring
\begin{itemize}
\item Differential reproductive success results in changes to gene frequencies
\end{itemize}
\end{itemize}
\end{itemize}
\section{09.18.20}
\label{sec:orga8c17b0}
\subsection{Hurricanes}
\label{sec:orgd180e9e}
\subsubsection{How Hurricanes Form}
\label{sec:org8d281f4}
\begin{itemize}
\item Water evaporates over the ocean and forms clouds when it touches cold air
\item A column of low pressure develops at the center with winds around the column
\item Speed of the wind around it increases
\end{itemize}
\begin{itemize}
\item Categorized based on wind speed (1-5)
\item Hurrican development requires warm water and low wind shear
\begin{itemize}
\item Carribean has warm water all year but also high wind shear which isn't conducive to hurricanes
\end{itemize}
\end{itemize}
\subsubsection{Climate Change \& Hurricanes}
\label{sec:orga23d8ee}
\begin{itemize}
\item Storm surge more dangerous (accoutns for 90\% of hurricane deaths)
\item 40\% increase with a 0.5 decree C inc in temperature
\item Increasing of North Atlantic hurricane season
\item Climate change is expected to shift the Bermuda high westward
\begin{itemize}
\item Bermuda High is a pressure system over the Atlantic
\item Has the ability to move hurricanes on the Atlantic
\end{itemize}
\end{itemize}
\subsubsection{Hurricane Harvey Intensification}
\label{sec:org966f862}
\begin{itemize}
\item Went from a tropical depression to a Cat 4 Hurricane in 57 hours
\item Soil in TX affected the amount of water maintained in the Earth
\item Huge economic impacts
\end{itemize}
\subsubsection{General Impacts}
\label{sec:orga387d2b}
\begin{itemize}
\item Storm Surge
\item Extreme Rainfall
\item Potential Wind Speed
\end{itemize}
\section{09.16.20}
\label{sec:org21cc43a}
\subsection{Heat Waves}
\label{sec:orgca3ce99}
\begin{itemize}
\item Heat extremes doubled in frequency from 1980-1999 to 2000-2019
\item Climate change affecting heat waves
\begin{itemize}
\item Shifting the frequency of hot and cold weather, heat waves are more frequent
\item Exacerbating heat inducing droughts, dry land leads to even hotter temps
\end{itemize}
\item Causes: Global warming ->
\begin{itemize}
\item Large scale global circulation change
\item Atmospheric Blocking increase
\item Air mass temp increase
\end{itemize}
\item Effects and Consequences
\begin{itemize}
\item Decreased human productivity
\item Increased tropical disease and death
\item Environmental racism
\item Crop productivity decreases
\item Lower biodiversity
\item Decreased water availability
\item Increased fire risk
\end{itemize}
\end{itemize}
\subsection{Wildfires}
\label{sec:org13966cb}
\begin{itemize}
\item Climate change is increasing the size, intensity, and frequency of wildfires
\item Wildfires create more cimate change through the increase of carbon expulsion through wildfires
\item Wildfires have global impacts due to smoke and temperature changes
\item Wildfire season has gotten longer due to climate change
\end{itemize}
\section{09.14.20}
\label{sec:orgc3a5b7f}
\subsection{Coriolis Effect}
\label{sec:org2523700}
\begin{itemize}
\item Deflection of an object's path due to the rotation of the Earth
\item North and south poles have different deflections of wind patterns
\item Little/no deflection at the equator
\end{itemize}
\subsection{Air circulation}
\label{sec:org7ce1507}
\begin{itemize}
\item Hottest air at the equator, moves north or south, cools, then comes back into equator
\end{itemize}
\subsubsection{Cells}
\label{sec:org346f860}
\begin{itemize}
\item Hadley cells: 0-30 degrees North and South
\item Ferrell Cell: 30-60 degrees North
\item Polar cells: North and South poles
\item Northeast and Southeast trade winds (remember directions!)
\item Westerlies: bring rain and precipitation
\end{itemize}
\subsection{Surface Ocean Currents}
\label{sec:orgd8f56fa}
\begin{itemize}
\item Ocean currents also affect the distribution of climates
\item Surface ocean currents generated by wind, Coriolis effect, heat, and continents
\item Heat redistribution from the Tropics
\begin{itemize}
\item Trade winds push warm surface waters west
\item Water reaches continents and flows north and south
\item water cools
\item Westerlies push cooler water east
\item Water reaches continents and flows to equator
\end{itemize}
\end{itemize}
\subsection{El Nino (Southern Oscillation)}
\label{sec:org96ec73e}
\begin{itemize}
\item Recurring climate pattern involving changes in the termperature of waters in the central
and eastern tropical Pacific Ocean.
\item The ocean and atmosphere can interact to affect climate
\begin{itemize}
\item Water in the eastern pacific warms up
\item Sea level pressure drops but rises in the W pacific
\item Trade winds weaken
\item Upwelling in the Pacific is reduced
\item Warmer waters - increased rainfall in Peru
\item Cooler waters, drought in Australia/Indonesia
\end{itemize}
\item Critical because of its ability to change atmospheric circulation, temps, and percipitation
\item Significantly hurts fisheries and developing countries
\end{itemize}
\subsection{La Nina}
\label{sec:orge95ed19}
\begin{itemize}
\item exacerbates normal conditions and leads to cooling in the Eastern pacific
\end{itemize}
\subsection{Heat Waves}
\label{sec:org84dfc37}
\begin{itemize}
\item Global warming has amplified the intensity, duration, and frequency of 
extreme heat and heat waves.
\end{itemize}
\section{09.11.20}
\label{sec:org315f8af}
\begin{itemize}
\item Northern latitudes experience greater seasonality in CO2 concentrations
\begin{itemize}
\item This is due to variation in photosynthetic activity by plants
\end{itemize}
\item Greenhouse effect
\begin{itemize}
\item Some incoming solar radiation is absorbed
\item Other amounts are reflected back into the atmosphere
\item Greenhouse gases capture and reradiate some heat over and over, warming the Earth
\item More gases, more heat
\end{itemize}
\item Albedo: measure of the reflectivity of a surface
\begin{itemize}
\item light surfaces have a higher albedo, darker surfaces have a lower albedo
\item surfaces with a low albedo release more heat into the atmosphere
\end{itemize}
\item Positive Feedback Loops
\begin{itemize}
\item applied to albedo:
\item temps rise -> more ice melting -> more water warming -> temps rise
\end{itemize}
\item Urban Heat Island Effect
\begin{itemize}
\item cities will be inc their population, inc energy and temperature
\item cities in particular have higher temperatures
\item tree cover -> cooler temperatures
\end{itemize}
\item Small changes in overall global temp can cause significant changes
in weather creating more extreme storms and more record temps
\begin{itemize}
\item roughly twice as many heat records
\item alterations in global jet streams
\item frost comes later and begins earlier
\end{itemize}
\item General climate change impacts:
\begin{itemize}
\item Health impacts
\item Crop productivity
\item Coastal erosion
\item Biodiversity
\item Water availability
\item Fire risk
\end{itemize}
\item Weather events getting more extreme with
\begin{itemize}
\item sea levels
\item wildfires
\end{itemize}
\item Need both adaptation and mitigation
\begin{itemize}
\item Adaptation: Responding to warming that has already happened
\item Mitigation: Preventing further warming by addressing climate change causes
\end{itemize}
\end{itemize}
\section{09.09.20}
\label{sec:org718a5f2}
\subsection{The Earth's Atmoshphere}
\label{sec:org5739fbc}
\begin{itemize}
\item Climate change is a serious environmental problem impacting species, ecosystems, and the globe
\item The atmosphere helps protect the Earth from the sun and keeps the temperature of the Earth cool
\item Atmosphere has a significant impact on climate
\item Earth's Atmosphere Composition
\begin{itemize}
\item Nitrogen (78\%)
\item Oxygen (21\%)
\item Other - Greenhouse Gases (1\%)
\end{itemize}
\end{itemize}
\subsection{The Keeling Curve}
\label{sec:org6968081}
\begin{itemize}
\item Curve developed to track atmospheric CO2 levels in Earth's atmosphere since 1952
\end{itemize}
\section{09.02.20}
\label{sec:org8585426}
\subsection{Demographic Transition Model}
\label{sec:org8e8df5e}
\begin{itemize}
\item Demographers use age structure diagrams to predict future growth potential of a population
\begin{itemize}
\item Pyramid structures indicate fast growth
\item House-shaped structures have moderate growth
\item Diamond structures have low/negative growth
\end{itemize}
\item Development leads to smaller families
\item Demographic transitions happen country by country
\item Industrialization might not lead to a demographic transition in all countries
\begin{itemize}
\item May not be linked to quality of life
\item Religion/Cultural beliefs
\item Social justice issue, improving the well-being of women and children key to dec. fertility
\end{itemize}
\end{itemize}
\subsection{Social Justice: Education for Women}
\label{sec:orgf3f77e3}
\begin{itemize}
\item Education of girls \& economic opportunities for women are correlated with lower birth rates
\item Education empowers women to take control over thri own fertility through: 
\begin{itemize}
\item Birth control
\item Marrying later
\item Delaying childbirth for career opportunities
\end{itemize}
\item Women earning more money is correlated to lower child mortality
\end{itemize}
\subsection{Environmental Impact}
\label{sec:orge6098fe}
\begin{itemize}
\item Slowing population growth is critical to sustainability and reducing our population impact
\item Our impact on the population is a result of (1) our population size and
(2) our consumption habits - both must be addressed
\item Ecological footprint: the land area needed to provide the resources for, and assimilate
the waste of, a person or population
\end{itemize}
\subsection{Sustainability}
\label{sec:org0da7ffd}
\begin{itemize}
\item A dynamic process between the economy, society, and environment
\item Sustainable: The process or the activity can be mantained without exhaustion or collapse
\begin{itemize}
\item Intra \& Inter-generational issue
\item Capacity of a system to accomodate changes:
\begin{itemize}
\item rates of renewable resource use should not exceed regeneration rate
\item rates of non-renewable resource use should not exceed rate of renewable substitute dev
\item rates of pollution should not exceed ssimilative capacity of the environment
\end{itemize}
\end{itemize}
\item Sustainable development has three factors:
\begin{itemize}
\item Social equity
\item Economic efficiency
\item Environmental responsibility
\end{itemize}
\end{itemize}
\subsection{Worldviews}
\label{sec:orgaec74ad}
\begin{itemize}
\item Culture influences our beliefs through:
\begin{itemize}
\item Knowledge
\item Beliefs
\item Values
\item Learned ways of life
\end{itemize}
\item Worldviews are affected by: 
\begin{itemize}
\item Environmental Ethics
\end{itemize}
\end{itemize}
\section{08.31.20}
\label{sec:orgd539fe6}
\subsection{Human Populations}
\label{sec:org07a12cb}
\begin{itemize}
\item 3 major sparks of growth
\begin{itemize}
\item Agricultural Revolution
\item Industrual Revolution
\item Green Revolution
\end{itemize}
\item With more food and technology, the population and need for more human labor increased
\item The human population is rapidly increasing and the impact of humans is due to:
\begin{itemize}
\item More humans overall
\item Greater growth / person
\end{itemize}
\item To address population growth, we need to pursue a variety of approaches that address factors
encouraging high birth rates
\item Zero population growth: the absence of population growth, occurs when birth rates = death rates
\begin{itemize}
\item Replacement fertility is reached
\end{itemize}
\end{itemize}
\subsection{Population Ecology}
\label{sec:org6eb393e}
\begin{itemize}
\item Analyze and categorize human populations using population ecology techniques
\item Population Ecology: a branch of biology dealing with the number of individuals
in a particular species in an area over time
\item Ecologists study populations to understand what makes them survive and thrive
\item Size, distribution, and growth rate is influenced by a variaty of factors and are important to 
understanding popilation ecology
\end{itemize}
\subsection{Monitoring Population Dynamics}
\label{sec:org58fdfee}
\begin{itemize}
\item Population Dynamics: Changes over time in population size and composition
\item Important metrics:
\begin{itemize}
\item Minimum viable population - min number of individuals that would still allow population to persist or grow
\item Carrying Capacity (K) - the maximum population size that a particular environment can support indefinitely
\end{itemize}
\item Population Density - the overall desnity a particular populaiton can sustain
\end{itemize}
\subsection{Exponential Growth \& Populations}
\label{sec:org8fdee76}
\begin{itemize}
\item Exponential growth occurs in populations when growth is unrestricted. This is, overall, unsustainable
\item Growth which becomes progressively larger each breeding cycle
\item Produces a J curve when plotted
\end{itemize}
\subsection{Monitoring Population Growth}
\label{sec:orgcf32773}
\begin{itemize}
\item Population growth rate - the rate at which a population of a species grows over time
\item Growth factors - factos which assist in the growth of a population
\item Resistance factors - factors which inhibit the growth of a population
\item Limiting factos: resources needed for survival but that may be in short supply
\end{itemize}
\subsection{Logistic Growth}
\label{sec:orgfefc96f}
\begin{itemize}
\item Occurs when a population nears carrying capacity (k) 
\begin{itemize}
\item Maximum sustainable population size
\item Determined by limiting factors
\end{itemize}
\end{itemize}
\subsection{Density-dependent/ Density-independent Factors}
\label{sec:org9122b17}
\begin{itemize}
\item Density dependent factors increase as populations grow, typically biotic
\begin{itemize}
\item Disease
\item Competition
\item Predation
\end{itemize}
\item Density independent facts affect population growth regardless of population size
\begin{itemize}
\item Storm
\item Fire/Flood
\item Avalanche
\end{itemize}
\end{itemize}
\subsection{Regulation}
\label{sec:org0efadee}
\begin{itemize}
\item Tendency for populations to decrease in size when above acertain level, and increase
in size below that level
\item Populations can only be regulated by density-dependent factors
\item Top down Regulation
\begin{itemize}
\item Predation
\item Disease
\end{itemize}
\item Bottom up Regulation
\begin{itemize}
\item Nutrients
\item Water
\item Sunlight
\end{itemize}
\end{itemize}
\section{08.28.20}
\label{sec:orgee97fa6}
\subsection{What is Science?}
\label{sec:orgd9be98f}
\begin{itemize}
\item Science: a body of knowledge that allows us to understand the world around us
\item Science is based on empirical evidence
\item Science allows us to test our ideas and evaluate the evidence
\item Scientific knowledge, including facts, theories, and laws, is subject to change
\item Scientific claims change as new evidence is made available
\end{itemize}
\subsection{White-Nose Syndrome Case Study}
\label{sec:org079c5a2}
\subsubsection{About WNS}
\label{sec:orgb0d14ac}
\begin{itemize}
\item White-Nose Syndrome
\begin{itemize}
\item 2007-2016, 6+ million bats dead as a result of White Nose Syndrome
\item The reason for the deaths was White-Nose Syndrome
\end{itemize}
\item Chytridiomycosis
\begin{itemize}
\item Infectious, fungal disease affecting amphibians
\item Helped understand white-nose syndrome with bats
\end{itemize}
\end{itemize}
\subsubsection{Science with WNS}
\label{sec:orgcb8879b}
\begin{itemize}
\item Scientific Method: the procedure used to empirically test a hypothesis
\begin{enumerate}
\item Observations generate questions
\item Choose a question to investigate
\item Consult literature
\item Develop a hypothesis and make a testable prediction
\item Design and carry out a study
\item Analyze data
\item Draw a conclusion
\end{enumerate}
\item Inferences: Conclusions drawn based on observations
\item Hypothesis: An inference that proposes possible explanation that includes previous knowledge/observation
\item Testing a Hypothesis: Hypotheses can be tested through an observational or experimental study
\item Scientific Studies: A fair test with results that could support or falsify the research prediction
\begin{itemize}
\item Experimental Studies: Conditions are manipulated intentionally
\begin{itemize}
\item Test Group: the group in an experimental study such that it differs from the control in only one way
\item Control Group: the group in an experimental study to which the test group's results are compared
\end{itemize}
\item Observational Studies: Gather real-world data without any intentional variable manipulation
\end{itemize}
\item Theory: A hypothesis that survives repeated testing by significant research can become a theory
\item Correlation v Causation
\begin{itemize}
\item Correlation: two things occuring together but not necessarily having a cause-effect relationship
\item Cause-Effect Relationship: the associationof a two variables that identifies one variable occurring
as a result of the other
\item Observational studies can derive correlation but not causation
\item Experimental studies can derive causational relationships
\end{itemize}
\item Policy: a formalized plan that addresses a desired outcome or goal
\begin{itemize}
\item policies need to be flexible, adapt to new findings, address the environmental problem, fit social need
and be economically viable in order to work effectively.
\end{itemize}
\end{itemize}
\subsection{Summary}
\label{sec:org81c1a4f}
\begin{itemize}
\item Scientific knowledge, through reliable and durable, is never absolute pr certain
\item This knowledge, including facts, theories, and laws, is subject to change
\item Physical evidence, systematically collected and logically analyzed, helps scientists
understand environmental issues and guide policy decisions
\end{itemize}
\section{08.25.20}
\label{sec:org4f85de2}
\subsection{Applied v Empirical Science}
\label{sec:org862928d}
\begin{itemize}
\item Applied Science = research whose findings are used to solve practical problems
\item Empirical science: A scientific approach that investigates the natural world through case studies
\end{itemize}
\subsection{Social Traps}
\label{sec:orge2f25b8}
\begin{itemize}
\item Occurs when a large amount of people are using a shared resource
\item Seem good in the short term but are actually bad in the long term
\item 3 Types:
\begin{itemize}
\item Tragedy of the Commons: When resources are shared, individuals try to maximize personal
benefit which hurts the resource itself
\item Time delay: Collective decisions that are good today but gone tomorrow
\item Sliding reinforcer: related to the evolution of natural organisms and GMOs
\end{itemize}
\end{itemize}
\subsection{Beginning with Data Interpretation}
\label{sec:org4a61d5d}
\begin{itemize}
\item Variables represent factors that can be manipulated, controlled, or merely measured for research
\item Variation = how much a variable changes
\item Independent var is controlled to see effects in the Dependent var
\item Graphs explore relationships with data and report this data
\end{itemize}
\subsection{Observational v Experimental Studies}
\label{sec:org2057776}
\begin{itemize}
\item Observational studies can observe a correlation but are unable to derive a causational reln.
\item Experimental studies have a control var (required) and are able to derive causactional rlns.
\end{itemize}
\section{08.24.20}
\label{sec:orgf16d0d0}
\subsection{Definitions}
\label{sec:org0151df5}
\begin{itemize}
\item Ecology: the branch of science dealing with the relationships of living things to one another \& the environment
\item Environmental Science: The study of all aspects of the environment, including physical, chemical, and biological factos, particularly with respect to how these aspects affect humans, and vice versa
\item Environmental Ethics: Personal philosophy that influences how a person interacts with their natural environment and thus influences how one responds to environmental problems
\end{itemize}
\subsection{Ecology != Environmentalism}
\label{sec:org431ecbf}
\begin{itemize}
\item Distinguish between envrironmentalism \& ecology
\end{itemize}

\begin{center}
\begin{tabular}{ll}
Environmentalism & Ecology\\
\hline
Activism to protect the environment & Scientific study of living and non-living things\\
\end{tabular}
\end{center}
\end{document}
