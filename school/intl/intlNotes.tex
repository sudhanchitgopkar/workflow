% Created 2020-09-08 Tue 11:33
% Intended LaTeX compiler: pdflatex
\documentclass[11pt]{article}
\usepackage[utf8]{inputenc}
\usepackage[T1]{fontenc}
\usepackage{graphicx}
\usepackage{grffile}
\usepackage{longtable}
\usepackage{wrapfig}
\usepackage{rotating}
\usepackage[normalem]{ulem}
\usepackage{amsmath}
\usepackage{textcomp}
\usepackage{amssymb}
\usepackage{capt-of}
\usepackage{hyperref}
\author{Sudhan Chitgopkar}
\date{\today}
\title{}
\hypersetup{
 pdfauthor={Sudhan Chitgopkar},
 pdftitle={},
 pdfkeywords={},
 pdfsubject={},
 pdfcreator={Emacs 26.3 (Org mode 9.1.9)}, 
 pdflang={English}}
\begin{document}

\tableofcontents

\section{09.02.20}
\label{sec:org4b22c06}
\subsection{State Development}
\label{sec:org96e811c}
\begin{itemize}
\item Europe v the New World
\begin{itemize}
\item Compare the state development of European, "old-world" countries and "new world" countries"
\begin{itemize}
\item Old world countries tend to be more imperialistic while new countries have a common exp
of being colonies
\item New world countries were composed of different types of people while 
Old world countries had a shared history
\end{itemize}
\end{itemize}
\item Feudalism: Geographic proximity and increasing power of feudal lords -> challenges between 
feudal properties were likely, so organization of resources and capabilities was key to survival
\item Feudalism led to increased collectivism, translating to:
\begin{itemize}
\item large, active labor organizations
\item large, state-provided social welfare
\item emphasis on production of higher quality goods instead of new innovation
\end{itemize}
\end{itemize}
\section{Module 3}
\label{sec:org2654f69}
\subsection{Institutions and States}
\label{sec:org76b55ff}
\subsubsection{Institutions}
\label{sec:orgeef7368}
\begin{itemize}
\item Institution: Institutions are formal and informal rules 
that structure the relationship among individuals
\item Can have legal or social forces
\item Institutions are resistant to change but can change as a 
\begin{itemize}
\item response to outside forces
\item response to internal pressures
\item response to effects of other institutions
\end{itemize}
\end{itemize}
\subsubsection{The State}
\label{sec:orge737763}
\begin{itemize}
\item An organization that maintains a legitimate monopoly of force over a certain territory
and its population
\item A set of political institutions sets policies for the territory and its population
\item Sovereignty: The ability for a state to carry out actions/policies within a territory
independently from external actors or internal rivals/challengers
\item Issues of autonomy and capcity: 
\begin{itemize}
\item Autonomy: the ability for the state to weild its power independently of the public
\item Capacity: the ability for the state to accrue and utilize sufficient resources to carry out
basic tasks and responsibilities
\end{itemize}
\end{itemize}
\subsubsection{Definitions}
\label{sec:orgc781367}
\begin{enumerate}
\item General
\label{sec:org5afbdac}
\begin{itemize}
\item State: governing structur's legitimate expression of sovereignty/main political organization 
of a country
\item Regime: Informal institutions that guide how a state operates
\item Government: Collection of actors in charge of carrying out political decisions of the regime
and in the interest of the state
\item Country: More generic; refers to the political collectivity of a soverieng territory
\item Nation: Refers to a group of people bound together by some trait who seek to establish 
to establish and express political interests
\item Nation != Country
\end{itemize}
\item Strength of States
\label{sec:orge425d78}
\begin{itemize}
\item Institutional Capabilities
\begin{itemize}
\item Strong States: Has good institutional foundations; these institutions function well
\item Weak States: Does not have good institutional foundations, its institutions do not function well
\item Failed States: Institutions so weak that they basically collapse and have no sovereignty
\end{itemize}
\item Organizational Structure
\begin{itemize}
\item Strong states maintain a fair amonut of centralized control
\item Weak states hand down authority to local institutions and are decentralized
\end{itemize}
\end{itemize}
\end{enumerate}
\subsection{Legitimacy \& Sovereignty}
\label{sec:orga01d248}
\begin{itemize}
\item Legitimacy: a value whereby something or someone is recognized and accepted by a large 
portion of the population as right and proper (is highly subjective)
\item Types of legitimacy:
\begin{itemize}
\item Traditional legitimacy: embodies historical myths/legends and continues from past to present
\item Charismatic legitimacy: Built on the force of ideas and appeals embodied by a leader
\item Rational-Legal legitimacy: Based on a system of laws and procedures that are institutionalized
\end{itemize}
\item Sources of Legitimacy:
\begin{itemize}
\item Conferred by the ruler to a ruler, government, or state
\item Ascribed to a state or ruler by other states or rulers (prerequisity for intl. cooperation)
\item Ascribed to a state or ruler by organizations/non-state actors
\end{itemize}
\item Legitimacy can often be used to push for change
\end{itemize}
\section{08.26.20}
\label{sec:org8de268b}
\subsection{Defining a Good Society}
\label{sec:orgc42d05d}
\begin{itemize}
\item Although observable, empirical assessments may differ from person to person,
depending upon factors that may distort individual observation.
\item Multiple factors contribute to whether a society is "good" or not, critical to comparing countries and 
political systems
\end{itemize}
\section{Module 2}
\label{sec:orgcc60f6f}
\subsection{Video 1}
\label{sec:orgf03cbc2}
\subsubsection{"Traditional Approach"}
\label{sec:org067eda8}
\begin{itemize}
\item Focus on a "formal-legal" aspects of political institutions
\item Mostly a categorizing exercise with little analysis
\item Many European ex-pats were these scholars
\end{itemize}
\subsubsection{Modern Era (1960s-1980s)}
\label{sec:orgeb5e9a1}
\begin{itemize}
\item Scholars stop describing, start comparing
\item Behavioral Revolution - emphasis on individual, group behavior, not static institutions
\item Gave rise to "developmentalism" or "modernization theory" 
\begin{itemize}
\item Proposed that a state develops economically, political and social development follows
\item Functionalism (functions of differently societal elements lay foundation for growth)
\end{itemize}
\end{itemize}
\subsubsection{Development (1960s-1980s)}
\label{sec:orgfd5e883}
\begin{itemize}
\item 5 stages each society goes through for development:
\item Traditional society (no mass production)
\item Preconditions for economic take-off (advent of industrialization and mass production)
\item Take-off (dynamic economic growth)
\item Drive to maturity (long era of econ growth, modern tech usage)
\item Age of high mass consumption (everyon is within driving distance of McDonalds (most places))
\end{itemize}
\subsubsection{Critiques of Behavioralims/Developmentalism}
\label{sec:org806f7a1}
\begin{itemize}
\item Enthocentric and ideologically driven
\item Creates dependency: capitalism creates a situation where underdeveloped countries depend
on developed countries
\item Developmentalist theories tried to be a one-size-fit-all theory which wasn't bale to be applied
to all individual case studies
\end{itemize}
\subsubsection{Post-Behavioralism (1990s-Present)}
\label{sec:org2b94989}
\begin{itemize}
\item Development of middle-range theories instead of one single theory
\item Diversity of approaches (qualitative, quantitative, case sudies)
\item Takes culture and historical context into consideration
\item Rational choice theory applied
\item Political economy: the state can have a varying role in economic matters
\end{itemize}
\subsubsection{New Institutionalism (Past 25 years)}
\label{sec:orgeecd663}
\begin{itemize}
\item Institutions are the nexus of political action
\item Institutions are dynamic that interact over time w other variables
\item Institutions comprise the surrounding environment \& sentiment
\end{itemize}

\subsection{Video 2}
\label{sec:orga63f764}
\subsubsection{The Study of Comparative Politics}
\label{sec:orgbd179d0}
\begin{itemize}
\item Comparative politics implies a method of study or an approach to an analysis, not a single theory
\item greatest challenge is that events occur in real time with unreplicable environments
\item events in politics can not be replicated to test for validity
\end{itemize}
\subsubsection{Goals}
\label{sec:orgb0c09f1}
\begin{itemize}
\item Goal: To assess which factors cause a certain outcome by comparing or contrasting cases
\item Cases: One of the group of things (events, states, actors, etc.) to be studied
\item Variable: a factor that changes over time or in different cases
\begin{itemize}
\item Independent var: causal var
\item Dependent var: outcome var
\end{itemize}
\item Causal relationships can be shown as:
\begin{itemize}
\item Cause -> effect
\item Independent var -> dependent var
\item Explanators var -> outcome
\item x var -> y var
\end{itemize}
\item Hypothesis: a possible answer that explains a causal effect
\end{itemize}
\subsubsection{Challenges}
\label{sec:orgf5628bb}
\begin{itemize}
\item Goal: to determine causality, not just correlation

\item In comparative politics, the researcher may not be able to:
\begin{itemize}
\item have a constant
\item measure certain variables
\item anticipate certain events
\item disentangle one variable from others
\item Access to cases \& information
\begin{itemize}
\item Langauage barriers
\item Time \& funding
\item Sufficient cases (and selection bias)
\item IRB (Institutional Review Board)
\end{itemize}
\end{itemize}
\item Correlation: when var A occurs with var B, one is not caused by the other
\item Endogeneity: when it cannot be determined whether an outcome was caused by another factor
or the outcome caused that factor to occur
\end{itemize}
\subsection{Video 3}
\label{sec:org14e5ebf}
\subsubsection{Most Similar Systems Design (MSS)}
\label{sec:orgf22acaf}
\begin{itemize}
\item A method in which as many independent vars as possible are held constant to explain a political
outcome: similar cases, different outcomes can help isolate a variable
\item Special Variation of MSS: Within-Case Comparison
\begin{itemize}
\item Single case analyzed over time or in different geographical areas
\item Breaks up a single case into subparts and allows for comparison
\end{itemize}
\end{itemize}
\subsubsection{Most-Different Systems Design (MDS)}
\label{sec:org0b3d968}
\begin{itemize}
\item Looks at cases that are different from one another and observes why the same political outcome is
observed as a method of understanding how to isolate a single causal variable
\end{itemize}
\subsubsection{Overview}
\label{sec:orgfe2af7c}
\begin{itemize}
\item Probable causal explanations (hypotheses): goal of these comparative approaches
\item Theories can be built from the strongest hypothesis
\item Theories can further be generalized based on the case
\end{itemize}
\end{document}