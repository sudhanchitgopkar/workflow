% Created 2020-09-03 Thu 14:35
% Intended LaTeX compiler: pdflatex
\documentclass[11pt]{article}
\usepackage[utf8]{inputenc}
\usepackage[T1]{fontenc}
\usepackage{graphicx}
\usepackage{grffile}
\usepackage{longtable}
\usepackage{wrapfig}
\usepackage{rotating}
\usepackage[normalem]{ulem}
\usepackage{amsmath}
\usepackage{textcomp}
\usepackage{amssymb}
\usepackage{capt-of}
\usepackage{hyperref}
\author{Sudhan Chitgopkar}
\date{\today}
\title{}
\hypersetup{
 pdfauthor={Sudhan Chitgopkar},
 pdftitle={},
 pdfkeywords={},
 pdfsubject={},
 pdfcreator={Emacs 26.3 (Org mode 9.1.9)}, 
 pdflang={English}}
\begin{document}

\tableofcontents

\section{Debating The Republic}
\label{sec:orge5d37e9}
\subsection{Socratics}
\label{sec:org2be54f0}
\subsubsection{Leadership Qualities}
\label{sec:org48ea492}
\begin{itemize}
\item Love of learning
\item Knowledge of one's own ignorace
\item Prioritizing state interests over individual ones
\end{itemize}
\subsubsection{Education}
\label{sec:orgc6f8688}
\begin{itemize}
\item Begins with understanding the arts, gentleness, and compassion
\item Followed by significant gymnastics
\item Education must be rooted in individual excellence
\item Not all leaders must be aristocrats, they simply need the proper education
\begin{itemize}
\item How does a non-aristocrat get such an education?
\end{itemize}
\item Payment for political participation is bad - one need not be incentivized for
participation and devotion to their state
\end{itemize}
\subsubsection{Citizenship}
\label{sec:org43451df}
\begin{itemize}
\item Anyone with the necessary aptitude, including women, can become citizens
\end{itemize}
\subsection{Thrasybulans}
\label{sec:org43c36e6}
\begin{itemize}
\item Injustice, while bad, indicates an unjust person rather than an unjust state
\item Education need not necessitate an artistic background - a military education is far more important
\item Socratic education is infeasible for all, which is unequal
\end{itemize}
\subsubsection{Citizenship}
\label{sec:orgbb175dc}
\begin{itemize}
\item Culture is critical to citizenship
\end{itemize}
\subsection{Solonians}
\label{sec:org9187eba}
\subsubsection{Leadership Qualities}
\label{sec:orgaacb87f}
\begin{itemize}
\item Leaders should be well-versed and acting in the best interest of the state
\item Leaders need to be well-rounded and certain people are better fit for these positions than others
\item The assembly is chaotic and ineffective as a means of decision-making and ruling
\end{itemize}
\subsubsection{Societal Qualities}
\label{sec:org365d9e6}
\begin{itemize}
\item Forgiveness is necessary for past wrong-doings
\item While wealth and education is largely cyclical, we should not be restructuring our society wholly
\item Metics and Low-income individuals should not have significant voices in assembly because they
don't have the education necessary to have a strong, educational conversation
\end{itemize}
\subsubsection{Citizenship}
\label{sec:orgf7628f0}
\begin{itemize}
\item Only strong, wealthy individuals should have citizenship to preserve the quality of Athens
\end{itemize}
\section{Characters \& Intro Notes}
\label{sec:org609cae5}
\subsection{Characters}
\label{sec:org5d8bb69}
\subsubsection{Assignments}
\label{sec:org3dc2f9a}
\begin{center}
\begin{tabular}{ll}
Names & Character\\
\hline
Tay & Lycon\\
Austin & Simon\\
Andrew & Aristachus\\
Natalie & Callias\\
Mac & Thrasybulus\\
Anjali & Lithicles\\
Pene;ope & Phlocles\\
Payton & Meletus\\
Dinah & Archinus\\
Jaylen & Lysimache\\
Grace & Aristocles\\
Catherine & Crito\\
Dylan & Lysias\\
Vegtri & Anytus\\
\end{tabular}
\end{center}

\subsection{Socrates \& Plato}
\label{sec:org9a6d4d2}
\subsubsection{Socrates}
\label{sec:org2e3686c}
\begin{itemize}
\item We have no texts by Socrates
\begin{itemize}
\item Texts from Plato, Xenophon, \& Aristophanes
\end{itemize}
\item "Founder of western philosophy
\item Taught through conversation
\begin{itemize}
\item Dialogie in agora, elsewhere in Athens
\end{itemize}
\end{itemize}
\subsubsection{Biography}
\label{sec:org3fc6867}
\begin{itemize}
\item Parents: Sophroniscus * Pharnarete
\item Personal life; three sons
\item No known profession
\item Military service: Potidaea, Amphipolis, Delium
\item Associated with the Thirty Tyrants (taught Critias)
\item Personal appearence: unkempt
\item Reputation in Athes: gafdly
\end{itemize}
\subsubsection{Plato}
\label{sec:orgf6d0ec0}
\begin{itemize}
\item Greek philosopher, mathematician, stident of socrates, wroter of philosophical dialogue
\item Founder of "The Academy"
\item Plato taught Aristotle
\item Large amount of works by Plato
\begin{itemize}
\item 36 dialogies (feat. Socrates and others)
\item 13 letters (may be by Plato)
\end{itemize}
\item Aristocratic famoly in Athens
\item Parents: Ariston (descendant of Athenian king) and Perictione (niece of Critias)
\end{itemize}
\subsubsection{Plato's Argumentation}
\label{sec:org23f3d34}
\begin{itemize}
\item Inductive reasoning: from particular examples to general truths
\item Deductive reasoning: from general truths to a particular example within the subset of that truth
\item Analogy: allows speakers to evoke in audience something they know and then apply its attributes
to somehting that is unfamiliar to them
\item Dialogue: Athenian public life is a matter of public debate/discussion/argument (Assembly)
\end{itemize}

\subsubsection{The Republic}
\label{sec:org82b7e87}
\begin{itemize}
\item Written 380-375 BCE but claims to record a conversation during the Peloponnesian War
\item Definition of justice and the role of a character in a just polis
\item Book 1: two definitions are proposed and rejected
\item Book 2: Flaucon's and Adeimantus' speeches \& definitions of justice
\end{itemize}
\end{document}